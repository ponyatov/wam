\clearpage
\secly{Foreword to the reprinted edition}

This document is a reprinted edition of the book bearing the same title that was
published by MIT Press, in 1991 under ISBN 0-262-51058-8 (paper) and ISBN
0-262-01123-9 (cloth). The MIT Press edition is now out of print, and its
copyright has been transferred to the author. This version is available for free
to anyone who wants to use it for \underline{non-commercial} purposes, from the
author’s web site at:

\medskip
\url{http://www.isg.sfu.ca/˜hak/documents/wam.html}
\medskip

\noindent
If you retrieve it, please let me know who you are, and for what purpose you
intend to use it.
Thank you very much.
\begin{flushright}
$H-A-K$\\
Burnaby, BC, Canada\\
May 1997
\end{flushright}

\secly{Предисловие}

\prolog\ был создан в 70х годах Аланом Колмерауером и его коллегами в
Университете Марселя. Это было первое практическое воплощение концепции
логического программирования, благодаря методам предложенным Робертом Ковальски.
Ключевая идея логического программирования заключается в том, что вычисления
могут быть выражены через управляемую дедукцию из декларативных утверждений.
Несмотря на то, что эта область значительно развилась с того времени, \prolog\
остается самым фундаментальным и широко используемым языком логического
программирования.

Первой реализацией \prologа был интерпретатор написанный на Фортране членами
группы Колмерауэра. Это была довольно грубая реализация, которая тем не менее
стала важной вехой во многих смыслах: она подтвердила жизнеспособность \prologа,
помогла распространению языка, и заложила основы технологии его реализации.
Следующим значимым этапом была возможно \prolog-система для DEC-10 разработанная
в Университете Эдинбурга мной\note{Warren}\ и моими коллегами. Эта система
использовала технику реализации Марсельской системы, с добавлением функционала
компиляции \prolog а в низкоуровневый язык\note{ассемблер DEC-10}, а также
несколько важных оптимизаций использования памяти. Позже я улучшил и
абстрагировал принципы реализации DEC-10 \prolog\ в то, что сейчас известно как
WAM: Warren Abstract Machine.

WAM это абстрактная (виртуальная) машина содержащая архитектуру памяти и набор
инструкций, адаптированные для \prologа. Она может быть эффективно реализована
на широком диапазоне компьютеров, и служит целевой плафтормой для портируемых
копиляторов \prolog а. Сейчас WAM стал считаться стандартом для реализаций
\prolog. Это приятно лично, но несколько смущает то что WAM слишком легко
принята в качестве стандарта. Несмотря на то, что WAM является дистилляцией
большого опыта в реализации \prolog а, это ни в коем случае не единственный
возможный вариант. Например WAM применяет ``копирование структуры'' для
представление \prolog-термов, в то время как ``разделение общих
структур" в Марсельской и DEC-10 реализациях имеет определенные преимущества,
и все еще может быть рекомендовано. Как бы то ни было, я считаю, что WAM,
безусловно, является хорошей отправной точкой для изучения технологии
реализации \prolog.

К сожалению, до сих пор не было хорошего источника для знакомства с WAM. Мой
первоначальный технический отчет не слишком доступен, содержит только ``голую``
спецификацию абстрактной машины, и написанной для читателя-эксперта. Другие
работы обсуждали WAM с разных точек зрения, но по-прежнему отсутствовало хорошее
учебное введение.

Поэтому очень приятно увидеть появление этого прекрасного учебника Хасана
A\^it-Kaci. Книгу приятно читать. Она объясняет WAM с большой ясностью и
элегантностью. Я считаю, что читатели, интересующиеся информатикой, найдут эту
книгу стимулирующем знакомством с увлекательным предметом\ --- реализацией
\prolog а. Я очень благодарен Хасану за то, что моя работа стала доступной для
более широкой аудитории.

\bigskip
\begin{flushright}
David H. D. Warren\\
Bristol, UK\\
February 1991
\end{flushright}
\clearpage

\secly{Acknowledgments}

\clearpage
\secly{Примечания переводчика}

Я давно интересуюсь применением алгоритмов AI\note{\emph{A}rtificial
\emph{I}ntelligence, приложения искуственного интеллекта} и
KRR\note{\emph{K}nowledge \emph{R}epresentation \& \emph{R}easoning, базы знаний
и экспертные системы} для разработки экспертных систем, баз знаний,
автоматизации синтеза аппаратно-програм\-мных систем\cite{bibilo},
трансформационного\note{\emph{T}ransformation \emph{P}rogramming} и
мета-программирования. В большинстве литературы по этой теме рекомендуется
использовать язык \prolog\ (или диалекты \lisp\ --- CLOS, CLIPS,
Sheme,\ldots).

К сожалению у языка \prolog\ совершенно непривычная методология программирования
и архитектура абстрактной машины, особенно для разработчиков, уже имеющих
некоторый опыт программирования на других в основном императивных языках:
\emph{чтобы писать на \prolog, нужно вывихнуть мозг}. Без понимания внутреннего
устройства языка это сделать крайне сложно.
С другой стороны, для практического применения необходимо не только понимать
работу \prolog-системы, но и при необходимости иметь возможность реализовать ее
самостоятельно, встроив в некоторый большой программный продукт, или запустить в
виде клиентского приложения в веб-браузере.

Поскольку мне более интересна фреймовая модель\cite{minsky} и использование
объектно-графовых баз данных как средства хранения, поиска и логического вывода,
возникает проблема реализации специализированной машины вывода по графу
представления знаний. В поисках книг по реализации таких систем удалось
найти эту книгу по устройству языка \prolog, и Yield
Prolog\note{\url{http://yieldprolog.sourceforge.net}}\ для языков,
поддерживающих генераторные функции: \py, \js\ и \csharp.

Поскольку книг по реализации языков программирования (особенно динамических
языков) на русском языке почти нет\cite{dragon,plai}, мне показалось полезным
сделать ``пиратский'' перевод для более глубокого понимания книги, и наработки
скиллов по чтению на английском. Если эта тема будет интересна, возможно стоит
подумать об официальном переводе и публикации на русском языке, или подготовки
отдельного адаптированного издания ``по мотивам'' (в зависимости от настроений
правообладателей:\\\copyright\ MIT Press, Hassan A\"it-Kaci).

\bigskip
\begin{description}
\item[github:] \url{https://github.com/ponyatov/wam}
\item[quora:] \url{http://ponyatov.quora.com/}
\item[email:] \email{dponyatov@gmail.com}
\end{description}
 