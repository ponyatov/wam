\clearpage
\secly{Foreword to the reprinted edition}

This document is a reprinted edition of the book bearing the same title that was
published by MIT Press, in 1991 under ISBN 0-262-51058-8 (paper) and ISBN
0-262-01123-9 (cloth). The MIT Press edition is now out of print, and its
copyright has been transferred to the author. This version is available for free
to anyone who wants to use it for non-commercial purposes, from the author’s web
site at:

\medskip
\url{http://www.isg.sfu.ca/˜hak/documents/wam.html}
\medskip

\noindent
If you retrieve it, please let me know who you are, and for what purpose you
intend to use it.
Thank you very much.
\begin{flushright}
$H-A-K$\\
Burnaby, BC, Canada\\
May 1997
\end{flushright}

\secly{Foreword}

\secly{Acknowledgments}

\clearpage
\secly{Примечания переводчика}

Я давно интересуюсь применением алгоритмов AI\note{\emph{A}rtificial
\emph{I}ntelligence, приложения искуственного интеллекта} и
KRR\note{\emph{K}nowledge \emph{R}epresentation \& \emph{R}easoning, базы знаний
и экспертные системы} для разработки экспертных систем, баз знаний,
автоматизации синтеза аппаратно-програм\-мных систем\cite{bibilo},
трансформационного\note{\emph{T}ransformation \emph{P}rogramming} и
мета-программирования. В большинстве литературы по этой теме рекомендуется
использовать язык \prolog\ (или диалекты \lisp\ --- CLOS, CLIPS,
Sheme,\ldots).

К сожалению у языка \prolog\ совершенно непривычная методология программирования
и архитектура абстрактной машины, особенно для разработчиков, уже имеющих
некоторый опыт программирования на других в основном императивных языках:
\emph{чтобы писать на \prolog, нужно вывихнуть мозг}. Без понимания внутреннего
устройства языка это сделать крайне сложно.
С другой стороны, для практического применения необходимо не только понимать
работу \prolog-системы, но и при необходимости иметь возможность реализовать ее
самостоятельно, встроив в некоторый большой программный продукт, или запустить в
виде клиентского приложения в веб-браузере.

Поскольку мне более интересна фреймовая модель\cite{minsky} и использование
объектно-графовых баз данных как средства хранения, поиска и логического вывода,
возникает проблема реализации специализированной машины вывода по графу
представления знаний. В поисках книг по реализации таких систем удалось
найти эту книгу по устройству языка \prolog, и Yield
Prolog\note{\url{http://yieldprolog.sourceforge.net}}\ для языков,
поддерживающих генераторные функции: \py, \js\ и \csharp.

Поскольку книг по реализации языков программирования (особенно динамических
языков) на русском языке почти нет\cite{dragon,plai}, мне показалось полезным
сделать ``пиратский'' перевод для более глубокого понимания книги, и наработки
скиллов по чтению на английском. Если эта тема будет интересна, возможно стоит
подумать об официальном переводе и публикации на русском языке, или подготовки
отдельного адаптированного издания ``по мотивам'' (в зависимости от настроений
правообладателей:\\\copyright\ MIT Press, Hassan A\"it-Kaci).

\bigskip
\begin{description}
\item[github:] \url{https://github.com/ponyatov/wam}
\item[quora:] \url{http://ponyatov.quora.com/}
\item[email:] \email{dponyatov@gmail.com}
\end{description}
 