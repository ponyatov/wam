\clearpage
\secly{Foreword to the reprinted edition}

This document is a reprinted edition of the book bearing the same title that was
published by MIT Press, in 1991 under ISBN 0-262-51058-8 (paper) and ISBN
0-262-01123-9 (cloth). The MIT Press edition is now out of print, and its
copyright has been transferred to the author. This version is available for free
to anyone who wants to use it for \underline{non-commercial} purposes, from the
author’s web site at:

\medskip
\url{http://www.isg.sfu.ca/˜hak/documents/wam.html}
\medskip

\noindent
If you retrieve it, please let me know who you are, and for what purpose you
intend to use it.
Thank you very much.
\begin{flushright}
$H-A-K$\\
Burnaby, BC, Canada\\
May 1997
\end{flushright}

\secly{Предисловие}

\prolog\ был создан в 70х годах Аланом Колмерауером и его коллегами в
Университете Марселя. Это было первое практическое воплощение концепции
логического программирования, благодаря методам предложенным Робертом Ковальски.
Ключевая идея логического программирования заключается в том, что вычисления
могут быть выражены через управляемую дедукцию из декларативных утверждений.
Несмотря на то, что эта область значительно развилась с того времени, \prolog\
остается самым фундаментальным и широко используемым языком логического
программирования.

Первой реализацией \prologа был интерпретатор написанный на Фортране членами
группы Колмерауэра. Это была довольно грубая реализация, которая тем не менее
стала важной вехой во многих смыслах: она подтвердила жизнеспособность \prologа,
помогла распространению языка, и заложила основы технологии его реализации.
Следующим значимым этапом была возможно \prolog-система для DEC-10 разработанная
в Университете Эдинбурга мной\note{Warren}\ и моими коллегами. Эта система
использовала технику реализации Марсельской системы, с добавлением функционала
компиляции \prolog а в низкоуровневый язык\note{ассемблер DEC-10}, а также
несколько важных оптимизаций использования памяти. Позже я улучшил и
абстрагировал принципы реализации DEC-10 \prolog\ в то, что сейчас известно как
WAM: Warren Abstract Machine.

WAM это абстрактная (виртуальная) машина содержащая архитектуру памяти и набор
инструкций, адаптированные для \prologа. Она может быть эффективно реализована
на широком диапазоне компьютеров, и служит целевой плафтормой для портируемых
копиляторов \prolog а. Сейчас WAM стал считаться стандартом для реализаций
\prolog. Это приятно лично, но несколько смущает то что WAM слишком легко
принята в качестве стандарта. Несмотря на то, что WAM является дистилляцией
большого опыта в реализации \prolog а, это ни в коем случае не единственный
возможный вариант. Например WAM применяет ``копирование структуры'' для
представление \prolog-термов, в то время как ``разделение общих
структур" в Марсельской и DEC-10 реализациях имеет определенные преимущества,
и все еще может быть рекомендовано. Как бы то ни было, я считаю, что WAM,
безусловно, является хорошей отправной точкой для изучения технологии
реализации \prolog.

К сожалению, до сих пор не было хорошего источника для знакомства с WAM. Мой
первоначальный технический отчет не слишком доступен, содержит только ``голую``
спецификацию абстрактной машины, и написанной для читателя-эксперта. Другие
работы обсуждали WAM с разных точек зрения, но по-прежнему отсутствовало хорошее
учебное введение.

Поэтому очень приятно увидеть появление этого прекрасного учебника Хасана
A\^it-Kaci. Книгу приятно читать. Она объясняет WAM с большой ясностью и
элегантностью. Я считаю, что читатели, интересующиеся информатикой, найдут эту
книгу стимулирующем знакомством с увлекательным предметом\ --- реализацией
\prolog а. Я очень благодарен Хасану за то, что моя работа стала доступной для
более широкой аудитории.

\bigskip
\begin{flushright}
David H. D. Warren\\
Bristol, UK\\
February 1991
\end{flushright}
\clearpage

\secly{Благодарности}

First and foremost, David H. D. Warren is the person to whom I must express
not only my awe for having invented and described the WAM, but also my most
sincere gratitude for never minding my repeatedly pestering him with questions
about minute details he had to dredge out from the deeps of his memory. I am all
the more indebted to him as I know how busy he never ceases being, designing
better and faster architectures, way ahead of most of us, no longer worrying about
this prehistoric part of his research life. In addition, I am particularly flattered that
he spontaneously cared to judge this humble opus to be a contribution worth being
part of the prestigious MIT Press Logic Programming Series. Finally, let him be
again thanked for granting me the honor of introducing it with a foreword.

To be honest, it is rather ironical that I, slow-witted as I notoriously am, be credited
with explaining the WAM to the world at large. In truth, my own deciphering of
the WAM’s intricacies has been a painful and lengthy process. As a matter of
fact, I owe my basic understanding of it to two dear friends, specifically. Thus,
I would like to thank Roger Nasr for introducing the WAM to me and Manuel
Hermenegildo for patiently explaining it to me a hundred times over. They deserve
most of the credit bestowed on me as this monograph’s author, having given me
the foundations upon which I structured my presentation.

This tutorial was, in an earlier version, a technical report of the Digital Equipment
Corporation’s Paris Research Laboratory (PRL). Several people contributed
to improve on the form and contents of that report and thence of this ensuing
monograph. Thus, I am very much in debt to Manuel Hermenegildo, David
Barker-Plummer, and David H. D. Warren, for the precious favor of proofreading
a first draft, making some important corrections. Many thanks are due also
to Patrick Baudelaire, Michel Gangnet, Solange Karsenty, Richard Meyer, and
Asc\'ander Su\'arez, for suggesting several emendations, having gracefully
volunteered to plow through the draft.

As the PRL report was being disseminated, I began receiving feedback from attentive
readers. Some of them caught a few serious bugs that remained in that
report making some material, as presented there, insidiously incorrect. Naturally,
all those mistakes have now been corrected in this monograph, and, where appropriate,
mention is made of those who brought to my attention my erroneous account.
Nevertheless, I would like to express here my gratitude to those who kindly
reported bugs, made insightful comments, gave disturbing counter-examples, or
proposed better explanations. They are: Christoph Beierle, Andr\'e Bolle,
Damian Chu, William Clocksin, Maarten van Emden, Michael Hanus, Pascal van Hentenryck,
Juhani Jaakola, Stott Parker, Fernando Pereira, Frank Pfenning, Dave
Raggett, Dean Rosenzweig, David Russinoff, and two anonymous reviewers. All
remaining mistakes are to be blamed on my own incompetence and still imprecise
understanding.

Having been presumptuous enough to envisage elaborating the original tutorial
into book form, I have benefitted from the kind advice and efficient assistance
of Bob Prior, editor at MIT Press. I thank him for everything—his patience in
particular.

Finally, I gratefully acknowledge the benevolent agreement kindly given to me by
Patrick Baudelaire, director of PRL, and Sam Fuller, Digital’s vice-president for
Corporate Research and Architecture, to let me publish the work as a book. I am
quite obliged for their laissez-faire.

\bigskip
\begin{flushright}
$H-A-K$\\
Rueil-Malmaison, France\\
January 1991
\end{flushright}

\clearpage
\secly{Примечания переводчика}

Я давно интересуюсь применением алгоритмов AI\note{\emph{A}rtificial
\emph{I}ntelligence, приложения искуственного интеллекта} и
KRR\note{\emph{K}nowledge \emph{R}epresentation \& \emph{R}easoning, базы знаний
и экспертные системы} для разработки экспертных систем, баз знаний,
автоматизации синтеза аппаратно-програм\-мных систем\cite{bibilo},
трансформационного\note{\emph{T}ransformation \emph{P}rogramming} и
мета-программирования. В большинстве литературы по этой теме рекомендуется
использовать язык \prolog\ (или диалекты \lisp\ --- CLOS, CLIPS,
Sheme,\ldots).

К сожалению у языка \prolog\ совершенно непривычная методология программирования
и архитектура абстрактной машины, особенно для разработчиков, уже имеющих
некоторый опыт программирования на других в основном императивных языках:
\emph{чтобы писать на \prolog, нужно вывихнуть мозг}. Без понимания внутреннего
устройства языка это сделать крайне сложно.
С другой стороны, для практического применения необходимо не только понимать
работу \prolog-системы, но и при необходимости иметь возможность реализовать ее
самостоятельно, встроив в некоторый большой программный продукт, или запустить в
виде клиентского приложения в веб-браузере.

Поскольку мне более интересна фреймовая модель\cite{minsky} и использование
объектно-графовых баз данных как средства хранения, поиска и логического вывода,
возникает проблема реализации специализированной машины вывода по графу
представления знаний. В поисках книг по реализации таких систем удалось
найти эту книгу по устройству языка \prolog, и Yield
Prolog\note{\url{http://yieldprolog.sourceforge.net}}\ для языков,
поддерживающих генераторные функции: \py, \js\ и \csharp.

Поскольку книг по реализации языков программирования (особенно динамических
языков) на русском языке почти нет\cite{dragon,plai}, мне показалось полезным
сделать ``пиратский'' перевод для более глубокого понимания книги, и наработки
скиллов по чтению на английском. Если эта тема будет интересна, возможно стоит
подумать об официальном переводе и публикации на русском языке, или подготовки
отдельного адаптированного издания ``по мотивам'' (в зависимости от настроений
правообладателей:\\\copyright\ MIT Press, Hassan A\"it-Kaci).

\bigskip
\begin{description}
\item[github:] \url{https://github.com/ponyatov/wam}
\item[quora:] \url{http://ponyatov.quora.com/}
\item[email:] \email{dponyatov@gmail.com}
\end{description}
 
\secrel{Введение}\secdown

В 1983 году David H.D.Warren разработал абстрактную (виртуальную) машину для
выполнения \prolog а, содержащую архитекутуру памяти и специализированный набор
команд \cite{War83}. Этот концепт стал широко известен как Warren Abstract
Machine (WAM), и стал стандартом де-факто для реализации \prolog-компиляторов.
В \cite{War83}\ Варрен описывает WAM в минималистоичном стиле, что делает
понимание очень сложным для рядового читателя, даже обладающего знаниями по
работе \prolog а. Слишком много остается невысказанным, и очень мало выражено в
ясных выражениях.\note{David H. D. Warren в частном порядке говорит что он
чувствовал что WAM имеет важное значение, но детали вряд ли представляли большой
интерес. Так что он использовал стиль личных заметок} Это привело к очень
небольшому количеству поклонников WAM, которые могли бы похвастаться пониманием
деталей своих разработок.  Как правило, это были разработчики Prolog, которые
решили постратить необходимое время, чтобы учиться через делание и кропотливо
достигать просветления.

\secrel{Existing literature}\secdown
\secup

\secrel{This tutorial}\secdown
\secrel{Disclaimer and motivation}

\secrel{Organization of presentation}

\secup

\secup
