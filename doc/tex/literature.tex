\secrel{Существующая литература}\secdown

Свидетельством этого недостатка понимания является тот факт, что за шесть лет
опубликовано мало публикаций, которые хотя бы бы обучали WAM, и уж тем более
формально доказывали ее правильность. Действительно, помимо оригинального
герметического отчета Варрена \cite{War83}, по WAM практически не было
официальных публикаций. Несколько лет назад можно было найти черновик,
составленный группой в Арагонской Национальной Лаборатории \cite{GLLO85}. Но
честно говоря мы обнаружили что эту рукопись еще сложнее понять чем сам отчет
Варрена. Надостаток заключался в том что черновик настаивал на представлении
полной WAM как есть, вместо постепенно трансформированного и оптимизированного
дизайна.

\clearpage
Немного улучшенный стиль был фактически использован David Maier и David S.
Warren\note{Это другой человек, а не разработчик WAM, для которого исследование
WAM было очень вдохновляющим. В свою очередь, достаточно интересно, что David H.
D. Warren в последнее время работал над параллельной архитектурой для \prolog а,
чья абстрактная модель разделяет некоторые идеи, независимо задуманные David S.
Warren}\ в \cite{MW88}. Там можно найти описание методов компиляции \prolog,
подобных WAM.\note{Op. Cit., Глава 11.}
Тем не менее, мы считаем, что это в любом случае весьма похвальное усилие
по-прежнему страдает от нескольких недостатков в качестве окончательного
учебника. Во-первых, он описывает близкий вариант WAM, а не, строго говоря, сам
WAM. То есть описаны не все функции WAM.
Более того, объяснения ограничиваются иллюстративными примерами и редко делают
явным и исчерпывающим образом конкретный контекст, в котором применяются
некоторые оптимизации.

Second, the part
devoted to compilation of Prolog comes very late in the book—in the penultimate
chapter—relying, for implementation details, on overly detailed Pascal procedures
and data structures incrementally refined over the previous chapters. We
feel that this sidetracks reading and obfuscates to-the-point learning of the abstract
machine. Finally, although it presents a series of gradually refined designs,
their tutorial does not separate orthogonal pieces of Prolog in the process. All the
versions presented are full Prolog machines. As a result, the reader interested in
picking and choosing partial techniques to adapt somewhere else cannot discriminate
among these easily. Now, in all fairness, Maier and Warren’s book has the
different ambition of being a first course in logic programming. Thus, it is actually
a feat that its authors were able to cover so much material, both theoretical
and practical, and go so far as to include also Prolog compiling techniques. More
important, their book is the first available official publication to contain a (real)
tutorial on the WAM techniques.

After the preliminary version of this book had been completed, another recent
publication containing a tutorial on the WAM was brought to this author’s attention.
It is a book due to Patrice Boizumault \cite{Boi88}\ whose Chapter 9 is devoted
to explaining the WAM. There again, its author does not use a gradual presentation
of partial Prolog machines. Besides, it is written in French—a somewhat restrictive
trait as far as its readership is concerned. Still, Boizumault’s book is very well
conceived, and contains a detailed discussion describing an explicit implementation
technique for the \emph{freeze} meta-predicate.\note{Op. Cit., Chapter 10.}

Even more recently, a formal verification of the correctness of a slight simplification
of the WAM was carried out by David Russinoff \cite{Rus89}. That work deserves
justified praise as it methodically certifies correctness of most of the WAM with
respect to Prolog’s SLD resolution semantics. However, it is definitely not a tutorial,
although Russinoff defines most of the notions he uses in order to keep his
work self-contained. In spite of this effort, understanding the details is considerably
impeded without working familiarity with the WAM as a prerequisite. At
any rate, Russinoff’s contribution is nevertheless a premi`ere as he is the first to
establish rigorously something that had been taken for granted thus far. Needless
to say, that report is not for the fainthearted.


\secup
